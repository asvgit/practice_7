\stepcounter{mysection}\section{\arabic{mysection} Программная реализация}

Однако, пусть реализация логического вывода на языке Prolog является целесообразной задачей, но для разделения моделируемой системы на логический блок и блок взаимодействия объектов необходима клиент-сервергая связка. А реализация сервера или клиента на языке Prolog не является его типовой задачей, что и касается реализации графической составляющей системы моделирования.

	Таким образом более целесообразном будет оставить блок взаимодействия объектов реализованными на языку Prolog.
		Более того, правила описанные для модуляции системы будут использованы
		для построения решения логическим блоком.

	Данный код иллюстрирует реализацию формулы реализующие обход возможных вариантов будущего, сбор статистики с каждого варианта и выбор наиболее подходящего варианта будущего по признаку. в данном случае интересующим признаком является среднее ожидание человеком кабины. 

\input{src/pl_mod_code.tex}

	Благодаря функционалу SWI-Prolog
		предикаты отражающие состояние системы в момент вызова кабины можно будет указывать в правилах только при необходимости.

\newpage
\stepcounter{mysection}\section{\arabic{mysection} Пример работы реализации}

	Данный пример иллюстрирует работу группы лифтов, где их количество равно 2, в здании с 5 этажами.
		В ходе работы системы появится два человека в моменты времени $t_2$ и $t_4$ на этажах $e_1$ и $e_0$,
		и у каждого человека целью будет четвёртый этаж $d_4$.

	Изначально первая кабина находится на первом этаже $e_0$, а вторя на втором $e_1$.
		Ниже приведён лог показывающий работу системы:
\begin{lstlisting}[basicstyle=\scriptsize]
24 May 2018 06:49:49 [71:null] TRACE Start modulation
24 May 2018 06:49:49 [71:null] WARNING Var 'Step' has null value
24 May 2018 06:49:49 [71:null] INFO Set first step '0'
24 May 2018 06:49:49 [71:0] TRACE Start step
24 May 2018 06:49:49 [71:0] TRACE Finish step
24 May 2018 06:49:49 [71:1] TRACE Start step
24 May 2018 06:49:49 [71:1] TRACE Finish step
24 May 2018 06:49:49 [71:2] TRACE Start step
24 May 2018 06:49:49 [71:2] INFO A man appears with id '0' on floor with id '1'
24 May 2018 06:49:49 [71:2] DEBUG Change list 'people_states' with id '0' res:'[1,0]'
24 May 2018 06:49:49 [71:2] TRACE Putting floor to map
24 May 2018 06:49:49 [71:2] DEBUG Current distances are '[1,0]'
24 May 2018 06:49:49 [71:2] DEBUG Current min dist '0' with id '1'
24 May 2018 06:49:49 [71:2] TRACE Copy var 'step' into 'r_0_step' with val '2'
24 May 2018 06:49:49 [71:2] TRACE Copy var 'people_targets' into 'r_0_people_targets' with val '[4,4]'
24 May 2018 06:49:49 [71:2] TRACE Copy var 'people_floors' into 'r_0_people_floors' with val '[1,0]'
24 May 2018 06:49:49 [71:2] TRACE Copy var 'people_waiting' into 'r_0_people_waiting' with val '[0,0]'
24 May 2018 06:49:49 [71:2] TRACE Copy var 'people_states' into 'r_0_people_states' with val '[1,0]'
24 May 2018 06:49:49 [71:2] TRACE Copy var 'elevators_floors' into 'r_0_elevators_floors' with val '[0,1]'
24 May 2018 06:49:49 [71:2] TRACE Copy var 'elev_rmap_1' into 'r_0_elev_rmap_1' with val '[]'
24 May 2018 06:49:49 [71:2] TRACE Copy var 'elev_people_1' into 'r_0_elev_people_1' with val '[]'
24 May 2018 06:49:49 [71:2] TRACE Copy var 'elev_rmap_0' into 'r_0_elev_rmap_0' with val '[]'
24 May 2018 06:49:49 [71:2] TRACE Copy var 'elev_people_0' into 'r_0_elev_people_0' with val '[]'
24 May 2018 06:49:49 [71:2] [r_0_:2] DEBUG Append to road map '[1,-1]' floor '1'
24 May 2018 06:49:49 [71:2] [r_0_:2] DEBUG Elev floors '[0,1]'
24 May 2018 06:49:49 [71:2] [r_0_:2] TRACE moving_elevators NElev '2' Id '0'
24 May 2018 06:49:49 [71:2] [r_0_:2] DEBUG Elevator on '0' has goal '1'
24 May 2018 06:49:49 [71:2] [r_0_:2] TRACE Change ElevPos '0' Id '0'
24 May 2018 06:49:49 [71:2] [r_0_:2] DEBUG Move elevator '0' from '0' to '1'
24 May 2018 06:49:49 [71:2] [r_0_:2] DEBUG Current elev '0' road map '[1,-1]' people '[]'
24 May 2018 06:49:49 [71:2] [r_0_:2] TRACE moving_elevators NElev '2' Id '1'
24 May 2018 06:49:49 [71:2] [r_0_:2] DEBUG Current elev '1' road map '[]' people '[]'
24 May 2018 06:49:49 [71:2] [r_0_:3] TRACE Start step
24 May 2018 06:49:49 [71:2] [r_0_:3] TRACE Do loop step
24 May 2018 06:49:49 [71:2] [r_0_:3] DEBUG A man with id '0' has been waiting for '0'
24 May 2018 06:49:49 [71:2] [r_0_:3] DEBUG Change list 'people_waiting' with id '0' res:'[1,0]'
24 May 2018 06:49:49 [71:2] [r_0_:3] DEBUG Elev floors '[1,1]'
24 May 2018 06:49:49 [71:2] [r_0_:3] TRACE moving_elevators NElev '2' Id '0'
24 May 2018 06:49:49 [71:2] [r_0_:3] DEBUG Elevator on '1' has goal '1'
24 May 2018 06:49:49 [71:2] [r_0_:3] DEBUG Elevator '0' has reached '1'
24 May 2018 06:49:49 [71:2] [r_0_:3] TRACE Manage people
24 May 2018 06:49:49 [71:2] [r_0_:3] INFO Going out people '[]' from elev '0'
24 May 2018 06:49:49 [71:2] [r_0_:3] TRACE Trace getting_people_in '2'
24 May 2018 06:49:49 [71:2] [r_0_:3] TRACE Trace getting_people_in '1'
24 May 2018 06:49:49 [71:2] [r_0_:3] DEBUG Change list 'people_states' with id '0' res:'[2,0]'
24 May 2018 06:49:49 [71:2] [r_0_:3] DEBUG Append to road map '[-1,4,-1]' floor '4'
24 May 2018 06:49:49 [71:2] [r_0_:3] INFO Coming people '[0]' to elev '0'
24 May 2018 06:49:49 [71:2] [r_0_:3] DEBUG Current elev '0' road map '[-1]' people '[0]'
24 May 2018 06:49:49 [71:2] [r_0_:3] TRACE moving_elevators NElev '2' Id '1'
24 May 2018 06:49:49 [71:2] [r_0_:3] DEBUG Current elev '1' road map '[]' people '[]'
24 May 2018 06:49:49 [71:2] [r_0_:3] TRACE Finish step
24 May 2018 06:49:49 [71:2] [r_0_:4] TRACE Start step
24 May 2018 06:49:49 [71:2] [r_0_:4] TRACE Do loop step
24 May 2018 06:49:49 [71:2] [r_0_:4] INFO A man appears with id '1' on floor with id '0'
24 May 2018 06:49:49 [71:2] [r_0_:4] DEBUG Change list 'people_states' with id '1' res:'[2,1]'
24 May 2018 06:49:49 [71:2] [r_0_:4] TRACE Putting floor to map
24 May 2018 06:49:49 [71:2] [r_0_:4] DEBUG Current distances are '[9,1]'
24 May 2018 06:49:49 [71:2] [r_0_:4] DEBUG Current min dist '1' with id '1'
24 May 2018 06:49:49 [71:2] [r_0_:4] TRACE Copy var 'step' into 'r_0_0_step' with val '4'
24 May 2018 06:49:49 [71:2] [r_0_:4] TRACE Copy var 'people_targets' into 'r_0_0_people_targets' with val '[4,4]'
24 May 2018 06:49:49 [71:2] [r_0_:4] TRACE Copy var 'people_floors' into 'r_0_0_people_floors' with val '[1,0]'
24 May 2018 06:49:49 [71:2] [r_0_:4] TRACE Copy var 'people_waiting' into 'r_0_0_people_waiting' with val '[1,0]'
24 May 2018 06:49:49 [71:2] [r_0_:4] TRACE Copy var 'people_states' into 'r_0_0_people_states' with val '[2,1]'
24 May 2018 06:49:49 [71:2] [r_0_:4] TRACE Copy var 'elevators_floors' into 'r_0_0_elevators_floors' with val '[1,1]'
24 May 2018 06:49:49 [71:2] [r_0_:4] TRACE Copy var 'elev_rmap_1' into 'r_0_0_elev_rmap_1' with val '[]'
24 May 2018 06:49:49 [71:2] [r_0_:4] TRACE Copy var 'elev_people_1' into 'r_0_0_elev_people_1' with val '[]'
24 May 2018 06:49:49 [71:2] [r_0_:4] TRACE Copy var 'elev_rmap_0' into 'r_0_0_elev_rmap_0' with val '[-1,4,-1]'
24 May 2018 06:49:49 [71:2] [r_0_:4] TRACE Copy var 'elev_people_0' into 'r_0_0_elev_people_0' with val '[0]'
24 May 2018 06:49:49 [71:2] [r_0_0_:4] DEBUG Append to road map '[-1,0,-1,4,-1]' floor '0'
24 May 2018 06:49:49 [71:2] [r_0_0_:4] DEBUG Elev floors '[1,1]'
24 May 2018 06:49:49 [71:2] [r_0_0_:4] TRACE moving_elevators NElev '2' Id '0'
24 May 2018 06:49:49 [71:2] [r_0_0_:4] DEBUG Elevator on '1' has goal '-1'
24 May 2018 06:49:49 [71:2] [r_0_0_:4] TRACE Elevator '0' is closing the door
24 May 2018 06:49:49 [71:2] [r_0_0_:4] TRACE Change ElevPos '1' Id '0'
24 May 2018 06:49:49 [71:2] [r_0_0_:4] DEBUG Move elevator '0' from '1' to '0'
24 May 2018 06:49:49 [71:2] [r_0_0_:4] DEBUG Current elev '0' road map '[0,-1,4,-1]' people '[0]'
24 May 2018 06:49:49 [71:2] [r_0_0_:4] TRACE moving_elevators NElev '2' Id '1'
24 May 2018 06:49:49 [71:2] [r_0_0_:4] DEBUG Current elev '1' road map '[]' people '[]'
24 May 2018 06:49:49 [71:2] [r_0_0_:5] TRACE Start step
24 May 2018 06:49:49 [71:2] [r_0_0_:5] TRACE Do loop step
24 May 2018 06:49:49 [71:2] [r_0_0_:5] DEBUG A man with id '1' has been waiting for '0'
24 May 2018 06:49:49 [71:2] [r_0_0_:5] DEBUG Change list 'people_waiting' with id '1' res:'[1,1]'
24 May 2018 06:49:49 [71:2] [r_0_0_:5] DEBUG Elev floors '[0,1]'
24 May 2018 06:49:49 [71:2] [r_0_0_:5] TRACE moving_elevators NElev '2' Id '0'
24 May 2018 06:49:49 [71:2] [r_0_0_:5] DEBUG Elevator on '0' has goal '0'
24 May 2018 06:49:49 [71:2] [r_0_0_:5] DEBUG Elevator '0' has reached '0'
24 May 2018 06:49:49 [71:2] [r_0_0_:5] TRACE Manage people
24 May 2018 06:49:49 [71:2] [r_0_0_:5] INFO Going out people '[]' from elev '0'
24 May 2018 06:49:49 [71:2] [r_0_0_:5] TRACE Trace getting_people_in '2'
24 May 2018 06:49:49 [71:2] [r_0_0_:5] DEBUG Change list 'people_states' with id '1' res:'[2,2]'
24 May 2018 06:49:49 [71:2] [r_0_0_:5] DEBUG Road map '[-1,4,-1]' has such floor '4'
24 May 2018 06:49:49 [71:2] [r_0_0_:5] TRACE Trace getting_people_in '1'
24 May 2018 06:49:49 [71:2] [r_0_0_:5] INFO Coming people '[1]' to elev '0'
24 May 2018 06:49:49 [71:2] [r_0_0_:5] DEBUG Current elev '0' road map '[-1,4,-1]' people '[0,1]'
24 May 2018 06:49:49 [71:2] [r_0_0_:5] TRACE moving_elevators NElev '2' Id '1'
24 May 2018 06:49:49 [71:2] [r_0_0_:5] DEBUG Current elev '1' road map '[]' people '[]'
\end{lstlisting}

Изучив выше изложенный журнал, можно увидеть, что происходит построение дерева формулы и её обход.
	Для того чтобы было нагляднее следует прокомментировать строку лога.

Первые четыре столбца в логе - это реальное время журналирования момента модуляции.
	Следующим столбцом идёт связка двух чисел, первое число - это номер процесса,
	он необходим для идентификации сессии, а второе число показывает момент времени модуляции.
	Ещё одним столбцом является связка строки и числа, число - это так же момент времени в данной ветки формулы,
	А строка отражает индекс чанка формулы в момент вывода, r означает корень выводимой формулы, а дальше серез нижние подчёркивание перечислены индексы кабин, которые участвуют в логическом выводе в данные момент.

